\documentclass{article}

% use Times
\usepackage{times}
% For figures
\usepackage{graphicx} % more modern
\graphicspath{ {img/} }
\usepackage{subfigure} 

% For citations
\usepackage{natbib}

% For algorithms
\usepackage{algorithm}
\usepackage{algorithmic}

% For hyperlinks
\usepackage{hyperref}
% Packages hyperref and algorithmic misbehave sometimes.  We can fix
% this with the following command.
\newcommand{\theHalgorithm}{\arabic{algorithm}}

% For mathematics
\usepackage{amsmath}
\usepackage{amsfonts}
\usepackage{amssymb}

\usepackage[accepted]{icml2015}

% The \icmltitle you define below is probably too long as a header.
% Therefore, a short form for the running title is supplied here:
\icmltitlerunning{Parallel Backpropagation for Multilayer Neural Networks}

\usepackage[top=0.8in, bottom=1in, left=0.8in, right=0.8in]{geometry}
%\usepackage[parfill]{parskip}

\begin{document} 

\twocolumn[
\icmltitle{Parallel Backpropagation for Multilayer Neural Networks}

% It is OKAY to include author information, even for blind
% submissions: the style file will automatically remove it for you
% unless you've provided the [accepted] option to the icml2015
% package.
\icmlauthor{Nitin Kamra}{nkamra@usc.edu}
\icmladdress{Department of Computer Science, University of Southern California}
\icmlauthor{Palash Goyal}{palashgo@usc.edu}
\icmladdress{Department of Computer Science, University of Southern California}
\icmlauthor{Sungyong Seo}{sungyons@usc.edu}
\icmladdress{Department of Computer Science, University of Southern California}
\icmlauthor{Vasileios Zois}{vzois@usc.edu}
\icmladdress{Department of Computer Science, University of Southern California}

% You may provide any keywords that you 
% find helpful for describing your paper; these are used to populate 
% the "keywords" metadata in the PDF but will not be shown in the document
\icmlkeywords{machine learning, neural network, backpropagation, gradient descent, parallel backpropagation}

\vskip 0.3in
]

\begin{abstract}

We present a parallel approach to classification using neural networks as the hypothesis class.
Neural networks can have millions of parameters and learning the optimum value of all parameters from huge datasets can be very time consuming task, if implemented on a single processor.
In this work, we have implemented parallel backpropagation to train Multi Layer Perceptrons (MLPs) for classification tasks. Specifically, we implement a serial backpropagation algorithm and its parallel counterpart with Pthread library in C++.
We also compare the two implementations with cuda implementation of backpropagation on a GPU and study the parallelization speedup obtained for various network architectures and increasing problem sizes.
We perform our tests on two benchmark datasets: MNIST and KDD Cup 1999, and finally compare all our implementations with the same implementations done with a state-of-the-art deep learning library: Theano.

\end{abstract} 

\section{Introduction}
\label{Intro}

Artificial neural networks are powerful machine learning tools used in many applications including but not limited to search engines, fraud detection, image classification, diagnostic medicine applications and stock market prediction.

Prior to application, neural networks undergo a training phase which is known to be very computationally intensive. This is primarily because the prevailing neural network architectures are implemented using several hidden layers, with each one consisting of thousands to millions of neurons in order to generalize well on diverse inputs. In this case, the resulting number of parameters that need to be trained are in the order of millions.

Furthermore, achieving high accuracy requires considering a large number of training examples (usually in the order of millions). For this reason training a neural network is both a data and resource intensive operation. This calls for efforts to parallelize the training process on multi-core machines and/or across multiple machines.

Currently Minibatch Gradient Descent (henceforth called MGD) is the most commonly used optimization algorithm used to train neural networks in supervised settings. It is implemented in a layerwise-recursive fashion which involves computing the neural network output (forward propagation) and updating parameters by computing gradient values (backpropagation).

In this paper, we have implemented parallel minibatch gradient descent to train multilayer feedforward neural networks for classification tasks.
The rest of the paper is organized as follows: section \ref{ProbDesc} presents a description of supervised learning tasks and neural networks as classifiers.
Section \ref{GD} describes the conventional gradient descent algorithm and its minibatch variant. It also describes the forward propagation and the backpropagation algorithms for neural networks.
Section \ref{ParGD} discusses approaches to parallelization: (a) by distributing multiple examples across several threads, and (b) by performing matrix computations in parallel.
It also discusses the implementation of parallel gradient descent on a GPU with CUDA.
Section \ref{Exp} describes our dataset and the experiments we have performed.
Section \ref{Results} presents our results obtained for these experiments and analyzes the speedup obtained for various network architectures and increasing problem sizes. It also presents a comparison with the same algorithms implemented using a state-of-the-art deep learning library Theano.
Section \ref{Concl} concludes the paper by discussing some potential applications and section \ref{Future} explores some potentially interesting future directions.

\section{Problem Description}
\label{ProbDesc}

We first describe the task of training feedforward neural networks for classification tasks using the conventional backpropagation algorithm. Feedforward neural networks act as function approximators in such tasks.

More formally, given a dataset $ \mathcal{D} = \{x_i,y_i\}_{i=1:N}$ with data points $x_i \in \mathbb{R}^D$ and labels $y_i \in \mathbb{R}^P$, the classification task involves making an accurate label prediction $\hat{y}$ on a previously unseen data point $x$. We approximate the label as a function of the datapoint using a classifier (a feedforward neural network here) with parameters $\theta = \{\theta_k\}_{k=1:K}$ as follows: $y \approx \hat{y} = f(x; \theta)$.
The classifier (neural network) learns the function $f$ from the training data $\mathcal{D}$ by tuning its weights ($\theta$) to minimize a pre-specified loss function, for instance, the Mean-Squared Error loss:
\begin{align}
\mathcal{L}_{MSE} (\theta) = \frac{1}{N}\sum_{i=1}^N ( y_i - f(x_i; \theta))^2
\end{align}

This minimization can be carried out using optimization algorithms like Gradient Descent, Newton's method, Levenberg-Marquardt algorithm etc.
Though Newton's method is a second-order optimization technique, it requires the computation of the hessian of the objective function which is very prohibitive for a large number of parameters like in a neural network.
Levenberg-Marquardt algorithm also requires computing matrices of the size of hessian and can be very slow for classifiers with a large number of parameters.
Currently Gradient Descent is the most successful technique to train huge neural networks with millions of parameters, since it provides a decent tradeoff between convergence speed and memory requirements.
We will describe gradient descent in section \ref{BackProp}. Now we describe a feedforward neural network.

The basic unit of feedforward neural networks is a neuron. A single neuron generally takes a vector of inputs $x \in \mathbb{R}^n$ and outputs a single scalar $y \in \mathbb{R}$. Generally the function computed by a neuron comprises of linear transformation on the input vector followed by a point-wise non-linear activation function:
\begin{align}
y = f(w^T x + b)
\end{align}
where $w \in \mathbb{R}^n$ is called the weight vector and $b \in \mathbb{R}$ is the scalar bias of the neuron.
Many different kinds of activation functions have been studied and are used according to the task at hand e.g. linear, sigmoid, tanh, ReLU etc. We will use the sigmoid and linear$(f(z) = z)$ activation functions in our implementation.
The sigmoid function is defined as follows:
\begin{align}
f(z) = \frac{1}{1 + e^{-z}}
\end{align}
A feedforward neural network is a directed acyclic graph $G = (V,E)$ each of whose vertices $v \in V$ is a neuron and every edge $(u,v) \in E$ represents the output of neuron $u$ going as an input to neuron $v$. In general to have a more concrete structure the graph $G$ is organized into a layered structure and we will only use layered feedforward neural networks in our implementations.

A feedforward network with $L$ layers comprises of a single input layer, $L-2$ hidden layers and a final output layer.
The $l^{th}$ layer has $N_l$ neurons and the full network has $N = \sum_{l=1}^L N_l$ neurons.

The first layer is the input layer and contains $N_1 = N_{in}$ dummy neurons each of which takes a single scalar input out of $N_{in}$ and passes it as it is.
The output from the input layer is a vector of size
\\
\\
Add details specific to neural nets


\section{Serial and Parallel Backpropagation}
\label{BackProp}

As explained in section \ref{ClassProb}. classifiers try to choose their parameters ($\theta$) in order to minimize some pre-specified loss function.
In this work, we will work with the Mean-squared Error loss function as defined in equation \ref{LMSE}.

\subsection{Gradient Descent}
\label{GD}

To optimize the loss function, we will use the Gradient Descent algorithm which in its most basic form takes the derivative of the cost function w.r.t. all the parameter values and updates the parameters by a value proportional to this gradient and opposite in sign.
In its most basic form the Gradient Descent algorithm takes the form shown in algorithm \ref{alg:GradDesc}.
\begin{algorithm}[tb]
   \caption{Gradient Descent}
   \label{alg:GradDesc}
\begin{algorithmic}
   \STATE {\bfseries Input:} data $x_i$, size $m$
   \REPEAT
   \STATE Initialize $noChange = true$.
   \FOR{$i=1$ {\bfseries to} $m-1$}
   \IF{$x_i > x_{i+1}$} 
   \STATE Swap $x_i$ and $x_{i+1}$
   \STATE $noChange = false$
   \ENDIF
   \ENDFOR
   \UNTIL{$noChange$ is $true$}
\end{algorithmic}
\end{algorithm}

\begin{enumerate}
\item Initialize all weights $(\theta)$ randomly with small values close to 0.
\item Repeat until convergence \{
\begin{equation*}
\theta_k := \theta_k - \alpha \frac{\partial \mathcal{L}_{MSE}}{\partial \theta_k} \hspace{16pt} \forall k \in \{1,2,...,K\}
\end{equation*}
\}
\end{enumerate}
Note that the derivative of the loss function generally takes the following form:
\begin{equation*}
\frac{\partial \mathcal{L}_{MSE}}{\partial \theta_k} = \frac{1}{N} \sum_{i=1}^N ( y_i - f(x_i; \theta)) \frac{\partial f(x_i; \theta)}{\partial \theta_k}
\end{equation*}
Since this requires a summation over all the training examples, the gradient computation is the biggest bottleneck for many supervised learning tasks on huge data sets.

On the other hand, having the gradient as a sum of partial gradients with respect to individual training examples opens up the possibility of parallelizing the gradient computation efficiently.
This can be done by distributing training examples across various processors/machines and letting them compute a partial gradient over their own training examples and then summing up these partial gradients to get the actual gradient.
A good approximation of this process can be obtained by using a mini-batch of training examples per iteration, instead of using the full dataset.

Describe backpropagation serial implementation in detail

Describe platform used \\
Describe backpropagation parallelization using pthreads


\section{Implementing BackPropagation on GPU}
\label{GPUBackProp}

Graphics Processing Units (GPUs) are massively parallel processors that were designed for efficient computer graphics rendering and image processing. In fact, GPUs are very effective when used to execute the graphics rendering pipeline which is a sequence of geometric transformations on small multidimensional vectors. For this reason, GPUs are suitable for executing very fast certain linear algebra operations including but not limited to vector-vector addition, matrix-vector and matrix-matrix multiplication. Moreover, because GPUs consist of many throughput oriented multiprocessors that follow the Single Instruction Multiple Data (SIMD) execution model, they can be very useful for applications that require processing and transformation operations on large dataset.

Training neural networks is a process that can be realized through a series of matrix-matrix multiplications and additions which are applied for a large number of training examples. Every neural network can be defined and operated on by following this abstraction. Back propagation can be implemented as a series of kernel executions that are combined to ultimately compute the change in the weight values caused by the corresponding training batch. In our implementation, a neural network is viewed as a collection of layers, each one consisting of 3 distinct matrices. These matrices store the outgoing weight values $W_i$, the incoming activation values $A_i$ and the delta error values $D_i$ computed by back propagation. The activation values of the input layer (i.e. $A_0$) are the actual training examples in the corresponding batch. The activation values and the delta error values are stored in column-major order. In contrast the weights for each layer are stored in row-major order. The bias values are embedded in the last column of the weight matrix.

We decompose back-propagation into 5 steps each one implemented by distinct kernel. The first step, handles the feed forward activation by implementing a tiled matrix-matrix multiplication based on Eq.~\ref{act_kernel}. Here we denote with $f$ the preferred activation function which is usually the sigmoid function. Our implementation is designed around templates and supports user defined activation functions. The feed-forward step is implemented using the tiled matrix-matrix multiplication as it is described in \cite{}. All threads participate in loading the tiles of the input matrices into shared memory. The threads use registers to accumulate partial results. After parsing the complete matrix each thread applies the activation function on the resulting value before storing it into global memory. All read and write operations to global memory are coalesced and there are no bank conflicts because thread iterate over the secondary matrix dimension in each case. An additional step is required to initialize the thread registers with the bias values.

\begin{equation}\label{act_kernel}
A_{i+1} = f\left(W_{i} \cdot A_{i} + b_i\right), \forall i \in \left[1,L-1\right]
\end{equation}

\begin{equation}\delta_output_kernel
D_{L} = Y - A_{L}
\end{equation}

\begin{equation}\label{delta_kernel}
D_{i} = W_{i}^T \cdot D_{i} \circ d(W_{i-1} \cdot A_{i+1})
\end{equation}

\begin{equation}\label{update_kernel}
W_{i} = W_{i} + \frac{n}{b} \cdot \sum_{j=1}^{b} D_{i+1}^{j} \cdot (A_{i}^j)^T
\end{equation}

Following this previous step, the delta error values are computed for each layer using Eq.~\ref{delta_kernel}. This equation is computed using 3 kernels, one for computing the output layer delta and two more that compute the transpose matrix-matrix multiplication and the hadamard product of the derivative for the corresponding activation function. A variation of the tiled matrix-matrix multiplication kernel is used to perform these operations. For the first kernel, the indexing is changed to enable multiplication with the transpose of the weight matrix. Here accesses to global memory remain coalesced since we order the thread blocks vertically on the weight matrix. However, we incur few bank conflicts when accessing the data in column-major order from shared-memory . In the second case, we chose to re-compute the product of $W_i \cdot A_i$ and multiply it with the result from the previous operation. This is because are goal was to support arbitrary defined activation functions. In case the sigmoid function is used, the derivative can be replaced with $A_i \cdot \left(1.0 - A_i\right)$ which enables significant reduction of the required computation.

Finally, we use the computed activation and the next layer delta matrices (Eq.~\ref{update_kernel}) to update the weights of the corresponding layer depending on the chosen learning rate and batch size. This operation can be considered as another variation of matrix-matrix multiplication where each column from $ D_{i+1}$ and $A_{i}$ are  multiplied to produced a single matrix. The number of resulting matrices are equal to the batch size and are accumulated in thread registers. The resulting summation is multiplied by $\frac{n}{b}$ and added to the weight values of the corresponding layer. For this operation, access to global memory is again coalesced. However, shared memory access incurs many bank conflicts which are proportional to the batch size. In order to improve the performance, the corresponding tiles are loaded transposed into shared memory. This incurs only a single bank conflict during loading and avoids many bank conflicts during the multiplication phase.

\section{Experiments}
\label{Exp}

Describe datasets and classification task \\
Decribe experimental network sizes \\
Describe hyperparameter selection \\
Describe experiments: serial vs pthreads vs cuda vs theano.

\section{Results and Analysis}
\label{Results}

All our networks give us about $93\%$ to $97.8\%$ classification accuracy on validation and test sets, but in this work we did not focus on getting a better classification accuracy.
Instead our major focus was to analyze the speedup obtained by parallelization, and the gigaflops of computation obtained for different batch sizes.

\begin{figure}[ht]
    \centering
    \begin{minipage}[t]{0.45\textwidth}
        \includegraphics[width=\textwidth]{net1h_batch_secs.png}
        \caption{Time per epoch for Net-1h}
		\label{fig:nn1_time}
    \end{minipage}
    \begin{minipage}[t]{0.45\textwidth}   
		\includegraphics[width=\columnwidth]{net1h_batch_gflops.png}
		\caption{GFLOPS for Net-1h}
		\label{fig:nn1_gflops}
    \end{minipage}
\end{figure}

\begin{figure}[ht]
    \centering
    \begin{minipage}[t]{0.45\textwidth}
        \includegraphics[width=\textwidth]{net2h_batch_secs.png}
        \caption{Time per epoch for Net-2h}
		\label{fig:nn2_time}
    \end{minipage}
    \begin{minipage}[t]{0.45\textwidth}   
		\includegraphics[width=\columnwidth]{net2h_batch_gflops.png}
		\caption{GFLOPS for Net-2h}
		\label{fig:nn2_gflops}
    \end{minipage}
\end{figure}

\begin{figure}[ht]
    \centering
    \begin{minipage}[t]{0.45\textwidth}
        \includegraphics[width=\textwidth]{net1h_speedup.png}
        \caption{Speedup for Net-1h}
		\label{fig:nn1_speedup}
    \end{minipage}
    \begin{minipage}[t]{0.45\textwidth}   
		\includegraphics[width=\columnwidth]{net2h_speedup.png}
		\caption{Speedup for Net-2h}
		\label{fig:nn2_speedup}
    \end{minipage}
\end{figure}

Figures \ref{fig:nn1_time} and \ref{fig:nn2_time} show bar plots of the time per epoch (in seconds, called TPE henceforth) by all implementations as the batch size increases.
Figures \ref{fig:nn1_gflops} and \ref{fig:nn2_gflops} show the corresponding bar plots for amount of computation performed per epoch (in gigaflops, called GFLOPS henceforth).
The speedups (called SUP henceforth) obtained for Net-1h and Net-2h are shown in figures \ref{fig:nn1_speedup} and \ref{fig:nn2_speedup} respectively for various batch sizes.

The following points can be immediately observed from the figures:
\begin{itemize}
\item TPE and GFLOPS are constant for serial implementations regardless of batch sizes, which is to be expected because serial implementations access all data sequentially regardless of batch size.
\item TPE decreases and GFLOPS increases with increase in batch size for parallel and GPU implementations, which parallelize on training examples. This is expected since having more training examples in a batch reduce the thread creation overheads.
\item GPUs have a huge number of parallel cores and threads per core and hence they clearly dominate over the multicore parallel implementations at all values of batch sizes and network configurations.
\item BLAS parallel implementations do not exhibit a change in TPE or GFLOPS with increasing batch sizes, since they do not parallelize on minibatches.
\item Multicore Parallel and CUDA GPU implementations give an average speedup of approximately 10 and 25 respectively.
\item Theano serial and BLAS serial implementations are very efficient already and hence their parallel counterparts demonstrate a smaller speedup, although the parallel counterparts parallelizing on matrix computations clearly win over the parallel implementations parallelizing on training examples, in terms of GFLOPS.
\item Theano implementations (serial, parallel and GPU) use both types of parallelization and hence demonstrate lower TPE than all their corresponding counterparts for large batch sizes where efficiency of parallelized matrix operations matters a lot.
\item For smaller batch sizes around 128, our CUDA implementation dominates over Theano GPU because parallelization of matrix computations have a larger overhead at such scales with smaller matrix sizes. This is also reflected in Theano GPU's speedup, which grows extremely fast as batch size grows (figures \ref{fig:nn1_speedup} and \ref{fig:nn2_speedup}).
\end{itemize}

\textbf{BLAS} As Basic Linear Algebra Subprograms, BLAS has been a crucial workhorse in heavy numerical computing.
By replacing our self-built LINALGLIB with BLAS, we obtained hugely improved results.
Note that the execution time of BLAS doesn't depend on the number of threads because it parallelizes each Matrix-vector product and hence each thread executes different parts of the same example.
The serial implementation is still a bit slower than Theano but is faster than our previous implementation by about 10 times.

\textbf{GPU Bottlenecks} Ideally  the number of overheads (allocation/reduction) should be monotonically reduced by increasing the size of the batch.
However, our experiment results show that there are some performance bottlenecks. In our implementation, shared memory bank conflicts which reduce the parallelism when threads access the shared memory appear when the number of batches increase.
When the dimensions (i.e. number of neurons between layers) of the neural network are not perfect multiples of 32 then some threads do not participate in the computation which results in degraded parallelism.

Overall, we demonstrate that the parallel computation is significantly faster than the serial computation if we utilize multiple threads for processing many examples or do matrix-vector computations in parallel.
By dividing training datasets into larger mini batches, we are able to perform better due to reduced parallelization overheads.

\section{Future Work}
\label{Future}

It would be an interesting idea to combine the two parallelization techniques that we have explored in this paper as future work.
Specifically, it might be possible to speed up the training even more by splitting training examples in parallel, and then further hierarchically parallelizing matrix computations for each individual example.

\bibliography{bibliography}
\bibliographystyle{icml2015}

\end{document} 

%\subsection{Figures}
%
%\begin{figure}[ht]
%\vskip 0.2in
%\begin{center}
%\centerline{\includegraphics[width=\columnwidth]{}}
%\caption{Historical locations and number of accepted papers for International
%  Machine Learning Conferences (ICML 1993 -- ICML 2008) and
%  International Workshops on Machine Learning (ML 1988 -- ML
%  1992). At the time this figure was produced, the number of
%  accepted papers for ICML 2008 was unknown and instead estimated.}
%\label{icml-historical}
%\end{center}
%\vskip -0.2in
%\end{figure} 
%
%\subsection{Algorithms}
%
%If you are using \LaTeX, please use the ``algorithm'' and ``algorithmic'' 
%environments to format pseudocode. These require 
%the corresponding stylefiles, algorithm.sty and 
%algorithmic.sty, which are supplied with this package. 
%Algorithm~\ref{alg:example} shows an example. 
%
%\begin{algorithm}[tb]
%   \caption{Bubble Sort}
%   \label{alg:example}
%\begin{algorithmic}
%   \STATE {\bfseries Input:} data $x_i$, size $m$
%   \REPEAT
%   \STATE Initialize $noChange = true$.
%   \FOR{$i=1$ {\bfseries to} $m-1$}
%   \IF{$x_i > x_{i+1}$} 
%   \STATE Swap $x_i$ and $x_{i+1}$
%   \STATE $noChange = false$
%   \ENDIF
%   \ENDFOR
%   \UNTIL{$noChange$ is $true$}
%\end{algorithmic}
%\end{algorithm}
% 
%\subsection{Tables} 
% 
%You may also want to include tables that summarize material. Like 
%figures, these should be centered, legible, and numbered consecutively. 
%However, place the title {\it above\/} the table with at least 
%0.1~inches of space before the title and the same after it, as in 
%Table~\ref{sample-table}. The table title should be set in 9~point 
%type and centered unless it runs two or more lines, in which case it
%should be flush left.
%
%% Note use of \abovespace and \belowspace to get reasonable spacing 
%% above and below tabular lines. 
%
%\begin{table}[t]
%\caption{Classification accuracies for naive Bayes and flexible 
%Bayes on various data sets.}
%\label{sample-table}
%\vskip 0.15in
%\begin{center}
%\begin{small}
%\begin{sc}
%\begin{tabular}{lcccr}
%\hline
%\abovespace\belowspace
%Data set & Naive & Flexible & Better? \\
%\hline
%\abovespace
%Breast    & 95.9$\pm$ 0.2& 96.7$\pm$ 0.2& $\surd$ \\
%Cleveland & 83.3$\pm$ 0.6& 80.0$\pm$ 0.6& $\times$\\
%Glass2    & 61.9$\pm$ 1.4& 83.8$\pm$ 0.7& $\surd$ \\
%Credit    & 74.8$\pm$ 0.5& 78.3$\pm$ 0.6&         \\
%Horse     & 73.3$\pm$ 0.9& 69.7$\pm$ 1.0& $\times$\\
%Meta      & 67.1$\pm$ 0.6& 76.5$\pm$ 0.5& $\surd$ \\
%Pima      & 75.1$\pm$ 0.6& 73.9$\pm$ 0.5&         \\
%\belowspace
%Vehicle   & 44.9$\pm$ 0.6& 61.5$\pm$ 0.4& $\surd$ \\
%\hline
%\end{tabular}
%\end{sc}
%\end{small}
%\end{center}
%\vskip -0.1in
%\end{table}
%
%Tables contain textual material that can be typeset, as contrasted 
%with figures, which contain graphical material that must be drawn. 
%Specify the contents of each row and column in the table's topmost
%row. Again, you may float tables to a column's top or bottom, and set
%wide tables across both columns, but place two-column tables at the
%top or bottom of the page.
% 
%\subsection{Citations and References} 
%
%Please use APA reference format regardless of your formatter
%or word processor. If you rely on the \LaTeX\/ bibliographic 
%facility, use {\tt natbib.sty} and {\tt icml2015.bst} 
%included in the style-file package to obtain this format.
%
%Citations within the text should include the authors' last names and
%year. If the authors' names are included in the sentence, place only
%the year in parentheses, for example when referencing Arthur Samuel's
%pioneering work \yrcite{Samuel59}. Otherwise place the entire
%reference in parentheses with the authors and year separated by a
%comma \cite{Samuel59}. List multiple references separated by
%semicolons \cite{kearns89,Samuel59,mitchell80}. Use the `et~al.'
%construct only for citations with three or more authors or after
%listing all authors to a publication in an earlier reference \cite{MachineLearningI}.
%
%Authors should cite their own work in the third person
%in the initial version of their paper submitted for blind review.
%Please refer to Section~\ref{author info} for detailed instructions on how to
%cite your own papers.
%
%Use an unnumbered first-level section heading for the references, and 
%use a hanging indent style, with the first line of the reference flush
%against the left margin and subsequent lines indented by 10 points. 
%The references at the end of this document give examples for journal
%articles \cite{Samuel59}, conference publications \cite{langley00}, book chapters \cite{Newell81}, books \cite{DudaHart2nd}, edited volumes \cite{MachineLearningI}, 
%technical reports \cite{mitchell80}, and dissertations \cite{kearns89}. 
%
%Alphabetize references by the surnames of the first authors, with
%single author entries preceding multiple author entries. Order
%references for the same authors by year of publication, with the
%earliest first. Make sure that each reference includes all relevant
%information (e.g., page numbers).
%
%\subsection{Software and Data}
%
%We strongly encourage the publication of software and data with the
%camera-ready version of the paper whenever appropriate.  This can be
%done by including a URL in the camera-ready copy.  However, do not
%include URLs that reveal your institution or identity in your
%submission for review.  Instead, provide an anonymous URL or upload
%the material as ``Supplementary Material'' into the CMT reviewing
%system.  Note that reviewers are not required to look a this material
%when writing their review.
%
%
%% Acknowledgements should only appear in the accepted version. 
%\section*{Acknowledgments} 
% 
%\textbf{Do not} include acknowledgements in the initial version of
%the paper submitted for blind review.
%
%If a paper is accepted, the final camera-ready version can (and
%probably should) include acknowledgements. In this case, please
%place such acknowledgements in an unnumbered section at the
%end of the paper. Typically, this will include thanks to reviewers
%who gave useful comments, to colleagues who contributed to the ideas, 
%and to funding agencies and corporate sponsors that provided financial 
%support.  
%
%
%% In the unusual situation where you want a paper to appear in the
%% references without citing it in the main text, use \nocite
%\nocite{langley00}